\documentclass[a4paper]{book}
\usepackage[times,inconsolata,hyper]{Rd}
\usepackage{makeidx}
\usepackage[utf8]{inputenc} % @SET ENCODING@
% \usepackage{graphicx} % @USE GRAPHICX@
\makeindex{}
\begin{document}
\chapter*{}
\begin{center}
{\textbf{\huge Package `PBF'}}
\par\bigskip{\large \today}
\end{center}
\begin{description}
\raggedright{}
\inputencoding{utf8}
\item[Type]\AsIs{Package}
\item[Title]\AsIs{A Novel Bayesian Design in Phase I Clinical Trial Based on Predictive Bayes Factor}
\item[Version]\AsIs{1.0.0}
\item[Author]\AsIs{Chenqi Fu, Xinying Fang, Shouhao Zhou}
\item[Maintainer]\AsIs{Xinying Fang }\email{fxy950225@gmail.com}\AsIs{}
\item[Imports]\AsIs{Iso, knitr}
\item[Description]\AsIs{The primary goal of phase I clinical trials is to find the maximum tolerated dose (MTD). To reach this objective, we introduce a new design for phase I clinical trials, the predictive Bayes factor (PBF) design. The PBF design is an innovative model-assisted design that is easy to implement in a manner similar to the traditional 3+3 design, as its decision rules can be pre-tabulated prior to the onset of trial, but is of more flexibility of selecting diverse target toxicity rates and cohort sizes. Moreover, it has satisfactory performance of selecting the true MTD to other more complicated model-based designs. We highlight the desirable properties of the PBF design, such as coherence and consistency. We conduct a numerical simulation study to compare the PBF design to the Bayesian optimal interval design. The results illustrate better performance of the PBF design, with respect to competitive average probabilities of choosing the MTD and lower risk treating patients at overly toxic doses.}
\item[License]\AsIs{GPL-2}
\item[Encoding]\AsIs{UTF-8}
\item[LazyData]\AsIs{true}
\end{description}
\Rdcontents{\R{} topics documented:}
